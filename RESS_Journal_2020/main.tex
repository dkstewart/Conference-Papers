%\documentclass{sig-alternate-05-2015}
\documentclass{llncs}
\pagestyle{headings}
\usepackage[margin=1.5in]{geometry}
\usepackage{makeidx}
\usepackage{tabularx,colortbl}
\usepackage[dvipsnames]{xcolor}
\usepackage{flushend}
\usepackage{cite}
\usepackage{amsmath}
\usepackage{amssymb}
\usepackage{epsfig}
\usepackage{stmaryrd}
\usepackage{url}
\usepackage{amsmath}
\usepackage{bm}
%\usepackage{amsthm}
\usepackage{amssymb}
\usepackage{multirow}
\usepackage{latexsym}
\usepackage{graphics}
\usepackage{graphicx}
\usepackage{enumitem}
\usepackage{comment}
\usepackage{longtable}
\usepackage{supertabular}
\usepackage{times}
\usepackage{listings}
\usepackage{subfigure}
\usepackage{color}
\usepackage{booktabs}
\usepackage{balance}
\usepackage{xspace}
\usepackage[ruled, vlined, linesnumbered]{algorithm2e}
\usepackage[autostyle]{csquotes}
\usepackage[]{algorithm2e}
\usepackage[tableposition=top]{caption}
\usepackage[nonumberlist]{glossaries}
%\usepackage[font=large]{caption}
%\newtheorem{theorem}{Theorem}
%\newtheorem{definition}{Definition}


%\newcommand{\definition}{\noindent \textbf{Definition} \citation{}}
%\newcommand{\theorem}{\noindent \textbf{Theorem} \citation{}}
%\newcommand{\lemma}{\noindent \textbf{Lemma} \citation{}}

%\newdef{lemma}{Lemma}
%\newdef{definition}{Definition}
%\newdef{theorem}{Theorem}
%\newdef{corollary}{Corollary}
%\newdef{note}{Note}
%\newdef{axiom}{Axiom}
\newcommand{\mkeyword}[1]{\mbox{\texttt{#1}}}
\DeclareMathOperator{\kuop}{uop}
\DeclareMathOperator{\kbop}{bop}
\DeclareMathOperator{\kite}{ite}
\DeclareMathOperator{\kpre}{pre}
\DeclareMathOperator{\dom}{dom}
\DeclareMathOperator{\ktrue}{true}
\DeclareMathOperator{\kfalse}{false}
\DeclareMathOperator{\kselect}{select}
\DeclareMathOperator{\ran}{range}
\newcommand{\lbb}{[\![}
\newcommand{\rbb}{]\!]}
\newcommand{\expr}{\phi}
\newcommand{\exprS}{\Phi}
\newcommand{\mats}[1]{\textcolor{red}{#1}}
\newcommand{\janet}[1]{\textcolor{blue}{#1}}
\newcommand{\darren}[1]{\textcolor{green}{#1}}
\newcommand{\danielle}[1]{\textcolor{orange}{#1}}

\sloppypar

\setacronymstyle{long-short}
\newacronym{aadl}{AADL}{Architecture Analysis and Design Language}
\newacronym{sae}{SAE}{Society of Automotive Engineering}
\newacronym{agree}{AGREE}{Assume Guarantee REasoning Environment}
\newacronym{mbse}{MBSE}{Model-based Systems Engineering}
\newacronym{mbsa}{MBSA}{Model-based Safety Analysis/Assessment}
\newacronym{arp}{ARP}{Aerospace Recommended Practices}
\newacronym{sfha}{SFHA}{System-level Functional Hazard Assessment}
\newacronym{pssa}{PSSA}{Preliminary System Safety Assessment}
\newacronym{ssa}{SSA}{System Safety Assessment}
\newacronym{fta}{FTA}{Fault Tree Analysis}
\newacronym{bsd}{BSD}{Berkley Software Distribution}
\newacronym{wbs}{WBS}{Wheel Brake System}
\newacronym{bscu}{BSCU}{Braking System Control Unit}
\newacronym{air}{AIR}{Aerospace Information Report}
\newacronym{hw}{HW}{Hardware}
\newacronym{sw}{SW}{Software}
\newacronym{cpu}{CPU}{Central Processing Unit}
\newacronym{osate}{OSATE}{Open Source AADL Tool Environment}
\newacronym{sasm}{SASM}{Safety Analysis System Model}
\newacronym{fem}{FEM}{Failure Effect Modeling}
\newacronym{esm}{ESM}{Existing System Model}
\newacronym{flm}{FLM}{Failure Logic Modeling}

\makenoidxglossaries


\begin{document}

\definecolor{gold}{rgb}{0.90,.66,0}
\definecolor{dgreen}{rgb}{0,0.6,0}
\newcommand{\stateequiv}{\equiv_{s}}
\newcommand{\traceequiv}{\equiv_{\sigma}}
\newcommand{\ta}{\text{TA}}
\newcommand{\cta}{\text{TA$_{C}$}}
\newcommand{\tta}{\text{TA$_{T}$}}
\newcommand{\ucalg}{\texttt{\small{IVC\_UC}}}
\newcommand{\ucbfalg}{\texttt{\small{IVC\_UCBF}}}


\title{AADL-Based Safety Analysis Using Formal Methods Applied to Aircraft Digital Systems}
%
\author{Danielle Stewart\inst{1}
\and Jing (Janet) Liu\inst{2}
\and Darren Cofer\inst{2}
\and Mats Heimdahl\inst{1}
\and \\ Michael W. Whalen\inst{1}
\and Michael Peterson\inst{3}}
\institute{University of Minnesota\\Department of Computer
Science and Engineering\\
\email{dkstewar, whalen, heimdahl}@cs.umn.edu
\and
Collins Aerospace\\
%Advanced Technology Center
Trusted Systems - Enterprise Engineering
\\
\email{Jing.Liu, Darren.Cofer}@collins.com
\and
Collins Aerospace\\
%Commercial Systems 
Flight Controls Safety Engineering - Avionics\\
\email{Michael.Peterson}@collins.com
}
\maketitle

\begin{abstract}
%Mats' abstract
Model-based development tools are increasingly being used 
for system-level development of safety-critical systems. Architectural 
and behavioral models  provide important information 
that can be leveraged to improve the system
safety analysis process. Model-based design artifacts 
produced in early stage development activities can be used to perform system safety analysis,
reducing costs, and providing accurate results throughout
the system life-cycle.  In this paper we describe an extension 
to the Architecture Analysis and Design Language (AADL) that 
supports modeling of system behavior under failure conditions. This 
\emph{Safety Annex} enables the independent modeling of component 
failures and allows safety engineers to weave various types of 
fault behavior into the nominal system model. The accompanying tool support uses model checking to verify safety properties in the presence of faults and comprehensively enumerate all applicable fault combinations leading to failure conditions
under quantitative objectives as part of the safety assessment process.  
We describe the Safety Annex, illustrate its use with a representative 
example, and discuss the tool support enabling an 
analyst to investigate the system behavior under failure conditions.	
	

\end{abstract}

\keywords{Model-based safety analysis; Model-based systems engineering; fault analysis; safety engineering; AADL; fault injection}

\input{introduction}

\input{preliminaries}
	%process
	%agree
	%updated_process
\section{Tool Architecture and Implementation}
\label{sec:implementation}

The Safety Annex is written in Java as a plug-in for the OSATE AADL toolset, which is built on Eclipse.  It is not designed as a stand-alone extension of the language, but works with behavioral contracts specified using the AGREE AADL annex~\cite{NFM2012:CoGaMiWhLaLu}. 
The architecture of the Safety Annex is shown in Figure~\ref{fig:plugin-arch}.

\begin{figure}
	\begin{center}
		%\includegraphics[trim=0 400 430 0,clip,width=0.85\textwidth]{images/arch.png}
		\includegraphics[width=\textwidth]{images/arch.png}
	\end{center}
	\vspace{-0.2in}
	\caption{Safety Annex Plug-in Architecture}
	\label{fig:plugin-arch}
	\vspace{-0.2in}
\end{figure}

AGREE contracts are used to define the nominal behaviors of system components as {\em guarantees} that hold when {\em assumptions} about the values the component's environment are met. When an AADL model is annotated with AGREE contracts and the fault model is created using the Safety Annex, the model is transformed through AGREE into a Lustre model~\cite{Halbwachs91:IEEE} containing the behavioral extensions defined in the AGREE contracts for each system component. 

%This program is intercepted by the Safety Annex plugin and fault model information is added in two ways, depending on which form of fault analysis is being run. 

%\subsection{Verification in the Presence of Faults}
When performing fault analysis, the Safety Annex extends the AGREE contracts to allow faults to modify the behavior of component inputs and outputs. An example of a portion of an initial AGREE node and its extended contract is shown in Figure~\ref{fig:lustre}. The left column of the figure shows the nominal Lustre pump definition is shown with an AGREE contract on the output; and the right column shows the additional local variables for the fault (boxes 1 and 2), the assertion binding the fault value to the nominal value (boxes 3 and 4), and the fault node definition (box 5). Once augmented with fault information, the AGREE model (translated into the Lustre dataflow language~\cite{Halbwachs91:IEEE}) follows the standard translation path to the model checker JKind~\cite{2017arXiv171201222G}, an infinite-state model checker for safety properties. 

\begin{figure}[h!]
	\hspace*{-2cm}
	\vspace{-0.3in} 
	\begin{center}
		%\includegraphics[trim=0 690 -10 70,clip,width=1.5\dimexpr\textwidth-2cm\relax]{images/lustre.pdf}
		\includegraphics[scale=0.3]{images/lustre.jpg}
		%\caption{Nominal AGREE node and its extension with faults}
		\caption{Nominal AGREE Node and Extension with Faults}
		\label{fig:lustre}
	\end{center}
	\vspace{-0.3in}
\end{figure}

There are two different types of fault analysis that can be performed on a fault model. The Safety Annex plugin intercepts the AGREE program and add fault model information to the model depending on which form of fault analysis is being run.

\textbf{Verification in the Presence of Faults}: This analysis returns one counterexample when fault activation per the fault hypothesis can cause violation of a property. The augmentation from Safety Annex to the AGREE program includes traceability information so that when counterexamples are displayed to users, the active faults for each component are visualized.

\textbf{Generate Minimal Cut Sets}: This analysis collects all minimal set of fault combinations that can cause violation of a property.
%\subsection{Generate Minimal Cut Sets}
Given a complex model, it is often useful to extract traceability information related to the proof, in other words, which portions of the model were necessary to construct the proof. An algorithm was introduced by Ghassabani, et. al. to provide Inductive Validity Cores (IVCs) as a way to determine which model elements are necessary for the inductive proofs of the safety properties for sequential systems~\cite{GhassabaniGW16}. Given a safety property of the system, a model checker can be invoked in order to construct a proof of the property. The IVC generation algorithm extracts traceability information from the proof process and returns a minimal set of the model elements required in order to prove the property. Later research extended this algorithm in order to produce all Minimal Inductive Validity Cores (All-MIVCs) to provide a full enumeration of all minimal set of model elements necessary for the inductive proofs of a safety property~\cite{Ghassabani2017EfficientGO}. 

\begin{comment}
The IVC algorithm considers a constraint system consisting of the assumptions and contracts of system components and the negation of the safety property of interest (i.e. the top level event). It then collects what are called Minimal Unsatisfiable Subsets (MUSs) of this constraint system; these are the minimal explanations of the constraint systems infeasibility in terms of the \textit{negation} of the safety property. Equivalently, these are the minimal model elements necessary to proof the safety property. In section \ref{sec:definitions}, we show the formal definitions of IVCs in detail. 

In order to compositionally generate minimal cut sets, the all IVC algorithm is used~\cite{Ghassabani2017EfficientGO}, but instead of annotating the model with IVC elements consisting of only assumptions and contracts, we insert the faults in each layer constrained to false. The leaf nodes contribute only constrained faults to the IVC elements as shown in Figure~\ref{fig:ivcElements1}. 
\end{comment}

In this approach, we use the all MIVCs algorithm to consider a constraint system consisting of the negation of the top level safety property, the contracts of system components, as well as the faults in each layer constrained to false. It then collects what are called Minimal Unsatisfiable Subsets (MUSs) of this constraint system; these are the minimal explanations of the constraint systems infeasibility in terms of the \textit{negation} of the safety property. Equivalently, these are the minimal model elements necessary to proof the safety property. In section \ref{sec:definitions}, we show the formal definitions in detail. The leaf nodes contribute only constrained faults to the IVC elements as shown in Figure~\ref{fig:ivcElements1}. 

In the non-leaf layers of the program, both contracts and constrained faults are considered as shown in Figure~\ref{fig:ivcElements2}. The reason for this is that the contracts are used to prove the properties at the next highest level and are necessary for the verification of the properties. 

The all MIVCs algorithm returns the minimal set of these elements necessary to prove the properties. This equates to any contracts or inactive faults that must be present in order for the verification of properties in the model. From here, we perform a number of algorithms to transform all MIVCs into minimal cut sets (see Section~\ref{sec:theory} for more details on the transformation algorithms).


\begin{figure}[h!]
	\hspace*{-2cm}
	\vspace{-0.1in} 
	\begin{center}
		\includegraphics[scale=0.5]{images/ivcElements1.png}
	\caption{IVC Elements used for Consideration in a Leaf Layer of a System}
		\label{fig:ivcElements1}
	\end{center}
\end{figure}

\begin{figure}[h!]
	\hspace*{-2cm}
	\vspace{-0.1in} 
	\begin{center}
		\includegraphics[scale=0.5]{images/ivcElements2.png}
	\caption{IVC Elements used for Consideration in a Middle Layer of a System}
		\label{fig:ivcElements2}
	\end{center}
\end{figure}


\begin{comment}
The architecture of the Safety Annex is shown in Figure~\ref{fig:plugin-arch}.  It is written in Java as a plug-in for the OSATE AADL toolset, which is built on Eclipse.  It is not designed as a stand-alone extension of the language, but works with behavioral contracts specified in AGREE AADL annex and associated tools~\cite{NFM2012:CoGaMiWhLaLu}.  AGREE allows {\em assume-guarantee} behavioral contracts to be added to AADL components.  The language used for contract specification is based on the Lustre dataflow language~\cite{Halbwachs91:IEEE}. AGREE improves scalability of formal verification to large systems by decomposing the analysis of a complex system architecture into a collection of smaller verification tasks that correspond to the structure of the architecture.

\begin{figure}
	\begin{center}
		%\includegraphics[trim=0 400 430 0,clip,width=0.85\textwidth]{images/arch.png}
		\includegraphics[width=.9\textwidth]{images/arch.png}
	\end{center}
	\vspace{-0.2in}
	\caption{Safety Annex Plug-in Architecture}
	\label{fig:plugin-arch}
\end{figure}

AGREE contracts are used to define the nominal behaviors of system components as {\em guarantees} that hold when {\em assumptions} about the values the component's environment are met.  The Safety Annex extends these contracts to allow faults to modify the behavior of component inputs and outputs.  To support these extensions, AGREE implements an Eclipse extension point interface that allows other plug-ins to modify the generated abstract syntax tree (AST) prior to its submission to the solver.  If the Safety Annex is enabled, these faults are added to the AGREE contract and, when triggered, override the nominal guarantees provided by the component.  

An example of a portion of an initial AGREE node and its extended contract is shown in Figure~\ref{fig:lustre}. 
In the left column of the figure, the nominal Lustre pump definition is shown with an AGREE contract on the output. In the right column, the additional local variables for the fault are seen in boxes 1 and 2, the assertion binding the fault value to the nominal value is seen in boxes 3 and 4, and the fault node definition is given in box 5. 

\begin{figure}[h!]
	\hspace*{-2cm}
	\vspace{-0.3in} 
	\begin{center}
		%\includegraphics[trim=0 690 -10 70,clip,width=1.5\dimexpr\textwidth-2cm\relax]{images/lustre.pdf}
		\includegraphics[scale=0.3]{images/lustre.jpg}
		\caption{Nominal AGREE node and its extension with faults}
		\label{fig:lustre}
	\end{center}
	\vspace{-0.3in}
\end{figure}


Once augmented with fault information, the AGREE model follows the standard translation path to the model checker JKind~\cite{2017arXiv171201222G}, an infinite-state model checker for safety properties.  The augmentation includes traceability information so that when counterexamples are displayed to users, the active faults for each component are visualized.

The architecture of the Safety Annex is shown in Figure~\ref{fig:plugin-arch}.  It is written in Java as a plug-in for the OSATE AADL toolset, which is built on Eclipse.  It is not designed as a stand-alone extension of the language, but works with behavioral contracts specified in AGREE AADL annex and associated tools~\cite{NFM2012:CoGaMiWhLaLu}.  AGREE allows {\em assume-guarantee} behavioral contracts to be added to AADL components.  The language used for contract specification is based on the Lustre dataflow language~\cite{Halbwachs91:IEEE}. AGREE improves scalability of formal verification to large systems by decomposing the analysis of a complex system architecture into a collection of smaller verification tasks that correspond to the structure of the architecture.

\begin{figure}
	\begin{center}
		%\includegraphics[trim=0 400 430 0,clip,width=0.85\textwidth]{images/arch.png}
		\includegraphics[width=.9\textwidth]{images/arch.png}
	\end{center}
	\vspace{-0.2in}
	\caption{Safety Annex Plug-in Architecture}
	\label{fig:plugin-arch}
\end{figure}

AGREE contracts are used to define the nominal behaviors of system components as {\em guarantees} that hold when {\em assumptions} about the values the component's environment are met.  The Safety Annex extends these contracts to allow faults to modify the behavior of component inputs and outputs.  To support these extensions, AGREE implements an Eclipse extension point interface that allows other plug-ins to modify the generated abstract syntax tree (AST) prior to its submission to the solver.  If the Safety Annex is enabled, these faults are added to the AGREE contract and, when triggered, override the nominal guarantees provided by the component.  

An example of a portion of an initial AGREE node and its extended contract is shown in Figure~\ref{fig:lustre}.  %The \texttt{\_\_fault} variables and declarations are added to allow the contract to override the nominal behavioral constraints (provided by guarantees) on outputs.  In the Lustre language, \texttt{assertion}s are constraints that are assumed to hold in the transition system. 
In the left column of the figure, the nominal Lustre pump definition is shown with an AGREE contract on the output. In the right column, the additional local variables for the fault are seen in boxes 1 and 2, the assertion binding the fault value to the nominal value is seen in boxes 3 and 4, and the fault node definition is given in box 5. 

%A  benefit of utilizing the AGREE behavioral annex is the ability to perform both monolithic and compositional analysis on the nominal model. AGREE allows {\em assume-guarantee} behavioral contracts to be added to AADL components.  The language used for contract specification is based on the Lustre dataflow language~\cite{Halbwachs91:IEEE} and the nominal model (AADL model annotated with AGREE contracts) is translated into Lustre before being sent to the JKind model checker for verification\cite{2017arXiv171201222G}. 

%When a user selects to run the fault analysis during verification, the AGREE contracts are automatically extended in Lustre in order to allow faults to modify the behavior of component outputs. These injections into the Lustre model are shown in Figure~\ref{fig:lustre}. 

\begin{figure}[h!]
	\hspace*{-2cm}
	\vspace{-0.3in} 
	\begin{center}
		%\includegraphics[trim=0 690 -10 70,clip,width=1.5\dimexpr\textwidth-2cm\relax]{images/lustre.pdf}
		\includegraphics[scale=0.3]{images/lustre.jpg}
		\caption{Nominal AGREE node and its extension with faults}
		\label{fig:lustre}
	\end{center}
	\vspace{-0.3in}
\end{figure}

\begin{comment}
\begin{figure}
	\vspace{-0.1in}
	%\includegraphics[trim=30 150 120 10,clip,width=\textwidth]{images/sample_code.png}
	\includegraphics[width=\textwidth]{images/sample_code.png}
	\vspace{-0.3in}
	\caption{Nominal AGREE node and its extension with faults}
	\label{fig:comp}
\end{figure}
\end{comment}
%An annotation in the AADL model determines the fault hypothesis.  This may specify either a maximum number of faults that can be active at any point in execution (typically one or two), or that only faults whose probability of simultaneous occurrence is above some probability threshold should be considered. In the former case, we assert that the sum of the true {\em fault\_\_trigger} variables is below some integer threshold.  In the latter, we determine all combinations of faults whose probabilities are above the specified probability threshold, and describe this as a proposition over {\em fault\_\_trigger} variables.
%
%With the introduction of dependent faults, active faults are divided into two categories: independently active (activated by its own triggering event) and dependently active (activated when the faults they depend on become active). The top level fault hypothesis applies to independently active faults. Faulty behaviors augment nominal behaviors whenever their corresponding faults are active (either independently active or dependently active).

%Once augmented with fault information, the AGREE model follows the standard translation path to the model checker JKind~\cite{2017arXiv171201222G}, an infinite-state model checker for safety properties.  The augmentation includes traceability information so that when counterexamples are displayed to users, the active faults for each component are visualized.


%For fault analysis, we separate the possible analyses available to users into two distinct actions and describe them here.


\section{Fault Modeling with the Safety Annex: Methodology}
\label{sec:fault_modeling}

To demonstrate the fault modeling capabilities of the Safety Annex we will use the \gls{wbs} described in \gls{air} 6110~\cite{AIR6110}.  This system is a well-known example that has been used as a case study for safety analysis, formal verification, and contract based design~\cite{DBLP:conf/cav/BozzanoCPJKPRT15, 10.1007/978-3-319-11936-6-7, CAV2015:BoCiGrMa, Joshi05:SafeComp}. This section first describes the \gls{wbs} system and then uses that system description to illustrate fault modeling using \gls{aadl}, \gls{agree}, and the safety annex. The analysis results of this examples will be described in Section~\ref{sec:fault_analysis_2}.

\subsection{Wheel Brake System Overview}
The preliminary work for the safety annex was based on a simple model of the \gls{wbs}~\cite{Stewart17:IMBSA}. To demonstrate a more complex fault modeling process, we constructed a functionally and structurally equivalent \gls{aadl} version of the more complex \gls{wbs} which was captured in NuSMV/xSAP models~\cite{DBLP:conf/cav/BozzanoCPJKPRT15}. Figure~\ref{fig:wbs} shows only one pair of wheels and their interactions with the rest of the system for clarity. The full version that was modeled in \gls{aadl} contains a total of 8 wheels.

\begin{figure}[h!]
	\centering
	%\includegraphics[trim=0 9 0 5,clip,width=\textwidth]{images/wbs_arch4_diagram.pdf}
	\includegraphics[width=\textwidth]{images/wbs_arch4.jpg}
	\caption{A Two-Wheel Diagram of the Wheel Brake System}
	\label{fig:wbs}
\end{figure} 

The \gls{wbs} is composed of two main parts: the control system and the electro-mechanical physical system. The physical system consists of redundant hydraulic circuits (designated green and blue) running from hydraulic pumps to wheel brakes as well as valves that control the hydraulic fluid flow. The physical system provides braking force to each of the eight wheels of the aircraft. The wheels are all mechanically braked in pairs. The control system commands electronic control of the physical system. %The \gls{bscu} of the control system electronically commands the physical system. 
The \gls{bscu} consists of two channels for redundancy in case a detectable fault occurs in the active channel. The \gls{bscu} also commands antiskid braking and controls the operating mode of the system through commands to the selector valve. These commands are sent to a selector valve component which selects which hydraulic pump supplies pressure depending on which operating mode the system is currently in. 

Top level inputs to the system include the mechanical pedal sensors and the power. These are considered black box components. The only pilot interaction modeled in this system is through mechanical braking command.

There are three operating modes in the WBS model:

\begin{itemize}
	\renewcommand{\labelitemi}{\textbullet}
	\item In \textit{normal} mode, the system uses the green hydraulic pump and one meter valve per each of the eight wheels (in Figure~\ref{fig:wbs}, this corresponds to e.g., ''Meter Valve (wheel 1)". Each of the meter valves are controlled through electronic commands coming from the active channel of the \gls{bscu}. These signals provide braking and antiskid commands for each wheel. The braking command is determined through a sensor on the pedal and the antiskid command is determined by the wheel sensors and detection of skid. 
	\item In \textit{alternate} mode, the system uses the blue hydraulic pump, four meter valves (one per wheel pair as shown in Figure~\ref{fig:wbs}: ''Meter Valve (Pair)"), and four antiskid shutoff valves (one per wheel pair). The meter valves are mechanically commanded through the pilot pedal corresponding to each wheel pair. If the selector detects lack of pressure in the green circuit, it switches to the blue circuit. Alternatively, if the \gls{bscu} detects a fault in the normal (green) mode of operation, the \gls{bscu} may likewise shut off the green pump and force a switch to alternate (blue) mode of operation. 
	\item \textit{Emergency} mode is triggered when the blue hydraulic pump fails. The accumulator component has a reserve of pressurized hydraulic fluid and will supply this to the blue circuit in emergency mode. 
\end{itemize}

The \gls{wbs} architecture model in \gls{aadl} contains 30 different kinds of components, 169 component instances, and a model depth of 5 hierarchical levels. 

\subsection{Nominal Model} 
The nominal, or behavioral, model is encoded using the \gls{agree} annex and the behavior is based on descriptions found in \gls{air}6110. The top level system properties are given by the requirements and safety objectives in \gls{air}6110. All of the subcomponent contracts support these system safety objectives through the use of assumptions on component input and guarantees on the output. The \gls{wbs} behavioral model in the \gls{agree} annex includes one top-level assumption and  11 top-level system properties, with 113 guarantees allocated to subsystems.  

An example system safety property is to ensure that there is no inadvertent braking of any of the wheels. This is based on a failure condition described in \gls{air}6110: \textit{Inadvertent wheel braking on one wheel during takeoff shall be less than 1E-9 per takeoff}. 
Inadvertent braking means that braking force is applied at the wheel but the pilot has not pressed the brake pedal.  In addition, the inadvertent braking requires that power and hydraulic pressure are both present, the plane is not stopped, and the wheel is rolling (not skidding). The property is stated in \gls{agree} such that inadvertent braking does \textit{not} occur, as shown in Figure \ref{fig:inadvertent_braking}. (The expression shown in Figure~\ref{fig:inadvertent_braking} \textit{true $\rightarrow$ property} in \gls{agree} is true in the initial state and then afterwards it is only true if property holds.)

\begin{figure}[h!]
	%\vspace{-0.2in}
	\begin{center}
		\includegraphics[width=.7\textwidth]{images/inadvertent_braking.png}
	\end{center}
	\vspace{-0.3in}
	\caption{AGREE Contract for Top Level Property: Inadvertent Braking}
	\label{fig:inadvertent_braking}
	%\vspace{-0.2in}
\end{figure}

%%%%%%%%%%%%							COMPONENT FAULT MODELING
\subsection{Component Fault Modeling}
\label{subsec:compFM}
The usage of the terms error, failure, and fault are defined in \gls{arp}4754A and are described here for ease of understanding~\cite{SAE:ARP4754A}. An \textit{error} is a mistake made in implementation, design, or requirements. A \textit{fault} is the manifestation of an error and a \textit{failure} is an event that occurs when the delivered service of a system deviates from correct behavior. If a fault is activated under the right circumstances, that fault can lead to a failure. The terminology used in EMV2 differs slightly for an error: an \textit{error} is a corrupted state caused by a \textit{fault}. The error propagates through a system and can  manifest as a \textit{failure}. In this report, we use the \gls{arp}4754A terminology with the added definition of \textit{error propagation} as used in EMV2. An \textit{error} is a mistake made in design or code and an \textit{error propagation} is the propagation of the corrupted state caused by an active \textit{fault}. 

The Safety Annex is used to add potential faulty behaviors to a component model. Within the \gls{aadl} component instance model, an annex is added which contain the fault definitions for the given component. The flexibility of the fault definitions allows the user to define numerous types of fault \textit{nodes} by utilizing the \gls{agree} node syntax. A library of common fault nodes has been written and is available in the project GitHub repository~\cite{SAGithub}. Examples of such faults include valves being stuck open or closed, output of a software component being nondeterministic, or power being cut off.  When the fault analysis requires fault definitions that are more complex, these nodes can easily be written and used in the model. 

When a fault is activated by its specified triggering conditions, it modifies the output of the component. This faulty behavior may lead to a violation of the contracts of other components in the system, including assumptions of downstream components. The impact of a fault is computed by the \gls{agree} model checker when the safety analysis is run on the fault model. 

As an illustration of fault modeling using the Safety Annex, we look at one of the components important to the inadvertent braking property: the brake pedal. When the mechanical pedal is pressed, a sensor reads this information and passes an electronic signal to the \gls{bscu} which then commands hydraulic pressure to the wheels. 

Figure~\ref{fig:sensor} shows the \gls{aadl} pedal sensor component with a contract for its nominal behavior. (The expression \textit{true $\rightarrow$ property} in \gls{agree} is true in the initial state and then afterwards it is only true if property holds.) The sensor has only one input, the mechanical pedal position, and one output, the electrical pedal position. 
A property that governs the behavior of the component is that the mechanical position should always equal the electronic position. 

\begin{figure}[h!]
	\hspace*{-2cm}
	%\vspace{-0.55in} 
	\begin{center}
		\includegraphics[trim=0 640 -10 70,clip,width=1.5\dimexpr\textwidth-2cm\relax]{images/system_sensor.pdf}
		\vspace{-0.3in}
		\caption{An AADL System Type: The Pedal Sensor}
		\label{fig:sensor}
	\end{center}
	\vspace{-0.2in}
\end{figure}

One possible failure for the pedal sensor is inversion of its output value. This fault can be triggered with probability $5.0\times 10^{-6}$ as described in \gls{air}6110 (in practice, the component failure probability is 
collected from hardware specification sheets).  
The Safety Annex definition for this fault is shown in Figure~\ref{fig:sensorFault}. Fault behavior is defined through the use of a fault node called \textit{inverted\_fail}.  When the fault is triggered, the nominal output of the component (\textit{elec\_pedal\_position}) is replaced with its failure value (\textit{val\_out}). 

\begin{figure}[h!]
	\hspace*{-2cm}
	%\vspace{-0.5in} 
	\begin{center}
		\includegraphics[trim=0 680 -10 70,clip,width=1.5\dimexpr\textwidth-2cm\relax]{images/safetyannex_sensorfault.pdf}
		\vspace{-0.2in}
		\caption{The Safety Annex for the Pedal Sensor}
		\label{fig:sensorFault}
	\end{center}
	\vspace{-0.2in}
\end{figure}

The \gls{wbs} fault model expressed in the Safety Annex contains a total of 33 fault definitions and 141 fault instances. The large number of fault instances is due to the redundancy in the system design and its replication to control 8 wheels.

%%%%%%%%%%%								IMPLICIT ERROR PROPAGATION
\subsection{Implicit Error Propagation}
\label{subsec:implicit}
In the Safety Annex approach, faults are captured as faulty behaviors that augment the system behavioral model in AGREE contracts. No explicit error propagation is necessary since the faulty behavior propagates through the nominal behavior contracts in the system model just as in the real system. The effects of any triggered fault are manifested through analysis of the \gls{agree} contracts. 

By contrast, in the \gls{aadl} Error Model Annex, Version 2 (EMV2)~\cite{EMV2} approach, all errors must be explicitly propagated through each component (by applying fault types on each of the output ports) for a component to have an impact on the rest of the system. To illustrate the key differences between implicit error propagation provided in the Safety Annex and the explicit error propagation provided in EMV2, we use a simplified behavioral flow from the WBS example using code fragments from EMV2, \gls{agree}, and the Safety Annex (Figure~\ref{fig:comparison_with_EMV2}). 

\begin{figure}[t]
	%\hspace*{-2cm}
	\vspace{-0.19in}
	\centering
	\includegraphics[trim=0 9 0 5,clip,width=\textwidth]{images/Comparison_with_EMV2.pdf}
	%\vspace{-0.3in}
	\caption{Differences between Safety Annex and EMV2}
	\label{fig:comparison_with_EMV2}
	%\vspace{-0.2in}
\end{figure} 

In this simplified \gls{wbs} system, the physical signal from the pedal component is detected by the sensor and the pedal position value is passed to the \gls{bscu} components.  The \gls{bscu} generates a pressure command to the valve component which applies hydraulic brake pressure to the wheels. 

In the EMV2 approach (top half of Figure~\ref{fig:comparison_with_EMV2}), the ``NoService'' fault is explicitly propagated through all of the components. These fault types are essentially tokens rather than a specifiction of the faulty behavior. At the system level, analysis tools supporting the EMV2 annex can aggregate the propagation information from different components to compose an overall fault flow diagram or fault tree. 

When a fault is triggered in the Safety Annex (bottom half of Figure~\ref{fig:comparison_with_EMV2}), the output behavior of the sensor component is modified. In this case the result is a ``stuck at zero'' error. The behavior of the \gls{bscu} receives a zero input signal and responds as if the pedal has not been pressed. This will cause the top level system contract to fail: {\em pedal pressed implies brake pressure output is positive}.

%%%%%%%%%%%%%%%								EXPLICIT ERROR PROPAGATION
\subsection{Explicit Error Propagation} 
\label{subsec:explicit}
Failures in \gls{hw} components can trigger behavioral faults in the system components that depend on them. For example, a \gls{cpu} failure may trigger faulty behavior in the threads bound to that \gls{cpu}. In addition, a failure in one \gls{hw} component may trigger failure in other \gls{hw} components located nearby, such as overheating, fire, or explosion
in the containment location. 
The Safety Annex provides the capability to explicitly model the impact of hardware failures on other faults, whether dependent or independent. The explicit propagation to non behavioral faults is similar to that provided in EMV2.

To better model faults at the system level that are dependent on \gls{hw} failures, a fault model element is introduced called a \textit{hardware fault}. Users are not required to specify behavioral effects for the \gls{hw} faults, nor are data ports necessary on which to apply the fault definition. An example of a model component fault declaration is shown below:
\begin{figure}[h!]
	\vspace{-0.1in}
	\begin{center}
	\includegraphics[width=.6\textwidth]{images/hw_fault2.png}
	\end{center}
	\vspace{-0.3in}
	\caption{Hardware Fault Definition}
	\label{fig:hwFault}
	%\vspace{-0.2in}
	\vspace{-0.2in}
\end{figure}

Users specify dependencies between the \gls{hw} component faults and faults that are defined in other components, either \gls{hw} or \gls{sw}. The hardware fault then acts as a trigger for dependent faults. This allows a simple propagation from the faulty \gls{sw} component to the \gls{sw} components that rely on it, affecting the behavior on the outputs of the affected \gls{sw} components.

In the \gls{wbs} example, assume that both the green and blue hydraulic pumps are located in the same compartment in the aircraft and an explosion in this compartment rendered both pumps inoperable. The \gls{hw} fault definition can be modeled first in the green hydraulic pump component as shown in Figure~\ref{fig:hwFault}. The activation of this fault triggers the activation of related faults as seen in the \textit{propagate\_to} statement shown in Figure~\ref{fig:hwFaultProp}. Notice that these pumps need not be connected through a data port in order to specify this propagation. 

\begin{figure}[h!]
	\vspace{-0.1in}
	\begin{center}
		\includegraphics[width=1.0\textwidth]{images/hw_prop_stmt.png}
	\end{center}
	\vspace{-0.3in}
	\caption{Hardware Fault Propagation Statement}
	\label{fig:hwFaultProp}
	%\vspace{-0.2in}
	%\vspace{-0.1in}
\end{figure}

The fault dependencies are specified in the system implementation where the system configuration that causes the dependencies becomes clear (e.g., binding between \gls{sw} and \gls{hw} components, co-location of \gls{hw} components). 


%%%%%%%%%%%%%								FAULT ANALYSIS STMTS
\subsection{Fault Analysis Statements}
\label{subsec:analysisStmts}
The fault analysis statement (also referred to as the fault hypothesis) resides in the \gls{aadl} system implementation that is selected for verification. This may specify the maximum number of faults that can be active at any point in execution (Figure~\ref{fig:hypothesisMaxN}).

\begin{figure}[h!]
	\vspace{-0.1in}
	\begin{center}
		\includegraphics[width=0.4\textwidth]{images/hypothesisMaxN.png}
	\end{center}
	\vspace{-0.1in}
	\caption{Max N Faults Analysis Statement}
	\label{fig:hypothesisMaxN}
\end{figure}
Alternatively, the fault analsis statement may specify that the only faults to be considered are those whose probability of simultaneous occurrence is above some probability threshold (Figure~\ref{fig:hypothesisProb}). 

\begin{figure}[h!]
	\vspace{-0.1in}
	\begin{center}
		\includegraphics[width=0.5\textwidth]{images/hypothesisProb.png}
	\end{center}
	\vspace{-0.1in}
	\caption{Probability Analysis Statement}
	\label{fig:hypothesisProb}
\end{figure}

Tying back to the fault tree analysis in traditional safety analysis, the former is analogous to restricting the cutsets to a specified maximum number of terms, and the latter is analogous to restricting the cutsets to only those whose simultaneous probability is above some set value. In the former case, we assert that the sum of the true {\em fault\_\_trigger} variables is at or below some integer threshold.  In the latter, we determine all combinations of faults whose probabilities are above the specified probability threshold, and describe this as a proposition over {\em fault\_\_trigger} variables. 

With the introduction of dependent faults, active faults are divided into two categories: independently active (activated by its own triggering event) and dependently active (activated when the faults they depend on become active). The top level fault hypothesis applies to independently active faults. Faulty behaviors augment nominal behaviors whenever their corresponding faults are active (either independently active or dependently active).

%%%%%%%%%%%%%%%%% 							ASYM FAULTS
\subsection{Asymmetric Faults and Implementation}
A \textit{Byzantine} or \textit{asymmetric} fault is a fault that presents different symptoms to different observers~\cite{Driscoll-Byzantine-Fault}. 
%In our modeling environment, asymmetric faults may be associated with a 
Consider a source component with an output that is connected to multiple inputs on different destination components. In this configuration, a \textit{symmetric} fault will result in all destination components observing the same faulty value from the source component. In an {\em asymmetric} fault, the destination components may observe different values from the source.  To capture the behavior of asymmetric faults %(``different symptoms to different observers''), 
it was necessary to extend our fault modeling mechanism in \gls{aadl}. 

To illustrate our implementation of asymmetric faults, assume a source component A has a 1-to-many output connected to four destination components (B-E) as shown in Figure~\ref{fig:commNodes} under ``Nominal System.'' If a symmetric fault was present on this output, all four connected components would see the same faulty behavior. An asymmetric fault should be able to present arbitrarily different values to the connected components. 

To this end, ``communication nodes'' are automatically inserted on each connection from component A to components B, C, D, and E (shown in Figure~\ref{fig:commNodes} under ``Fault Model Architecture''). From the users perspective, the asymmetric fault definition is associated with component A's output and the architecture of the model is unchanged from the nominal model architecture. Behind the scenes, these communication nodes are created to facilitate potentially different fault activations on each of these connections. The fault definition used on the output of component A will be inserted into each of these communication nodes as shown by the red circles at the communication node output in Figure~\ref{fig:commNodes}.
\begin{figure}[!htb]
        \center{\includegraphics[width=0.8\textwidth] {images/commNodes.png}}
        \caption{\label{fig:commNodes} Communication Nodes in Asymmetric Fault Implementation}
\end{figure}

\begin{figure}[!htb]
        \center{\includegraphics[width=\textwidth] {images/asymFaultDef.png}}
        \caption{\label{fig:asymFaultDef} Asymmetric Fault Definition in the Safety Annex}
\end{figure}

An asymmetric fault is defined for component A as in Figure~\ref{fig:asymFaultDef}. This fault defines an asymmetric failure on component A that when active, is stuck at a previous value (\textit{prev(Output, 0)}). This can be interpreted as the following: some connected components may only see the previous value of component A output and others may see the correct (current) value when the fault is active. This fault definition is injected into the communication nodes and which of the connected components see an incorrect value is completely nondeterministic. Any number of the communication node faults (0…all) may be triggered upon activation of the main asymmetric fault on the source output.





\section{Analysis of the Model}
\label{sec:fault_analysis_2}
In this section we describe results from the nominal model analysis and the fault analysis.  

\subsection{Nominal Model Analysis}
Before performing fault analysis, users should first check that the safety properties are satisfied by the nominal design model. This analysis can be performed monolithically or compositionally in AGREE. Using monolithic analysis, the contracts at the lower levels of the architecture are flattened and used in the proof of the top level safety properties of the system. Compositional analysis, on the other hand, will perform the proof layer by layer top down, essentially breaking the larger proof into subsets of smaller problems. For a more comprehensive description of these types of proofs and analyses, see additional publications related to AGREE \cite{cofer2012compositional,QFCS15:backes} 

The WBS has a total of 13 safety properties at the top level that are supported by subcomponent assumptions and guarantees. These are shown in Table \ref{tab:safetyProperties}. Given that there are 8 wheels, contract S18-WBS-0325-wheelX is repeated 8 times, one for each wheel. The behavioral model in total consists of 36 assumptions and 246 supporting guarantees.

\begin{table}[]
\begin{tabular}{@{}ll}
\toprule
\textbf{S18-WBS-R-0321} \\Loss of all wheel braking during landing or RTO shall be less than $5.0 \times 10^{-7}$ per flight.                                    \\ \midrule 
\textbf{S18-WBS-R/L-0322}  \\ Asymmetrical loss of wheel braking (Left/Right) shall be less than $5.0 \times 10^{-7}$ per flight. \\ \midrule
\textbf{S18-WBS-0323} \\ Never inadvertent braking with all wheels locked shall be less than $1.0 \times 10^{-9}$ per takeoff.                                                                                                                                                                                                               \\ \midrule
\textbf{S18-WBS-0324}  \\ Never inadvertent braking with all wheels shall be less than $1.0 \times 10^{-9}$ per takeoff.                                                                                                            \\ \midrule
\textbf{S18-WBS-0325-wheelX} \\ Never inadvertent braking of wheel X shall be less than $1.0 \times 10^{-9}$ per takeoff.                                                                                                           .                                                                                                                 \\ \bottomrule
\end{tabular}
\caption{Safety Properties of WBS}
\label{tab:safetyProperties}
\end{table}  

\subsection{Fault Model Analysis}
There are two main options for fault model analysis using the Safety Annex. The first option injects faulty behavior allowed by faulty hypothesis into the AGREE model and returns this model to JKind for analysis. This allows for the activity of faults within the model and traceability information provides a way for users to view a counterexample to a violated contract in the presence of faults. The second option is used to generate minimal cut sets for the model. The model is annotated with fault activation that are constrained to false as well as intermediate level guarantees as model elements for consideration for the all Minimal Inductive Validity Cores (All-MIVCs)
algorithm. The All-MIVCs traces the minimal set of model elements used to produce minimal cut sets and is described in Section~\ref{sec:theory}. This subsection presents these options and discusses the analytical results obtained. 

\subsubsection{Verification in the Presence of Faults: Max N Analysis}
Using a max number of faults for the hypothesis, the user can constrain the number of simultaneously active faults in the model. The faults are added to the AGREE model for the verification. Given the constraint on the number of possible simultaneously active faults, the model checker attempts to prove the top level properties given these constraints. If this cannot be done, the counterexample provided will show which of the faults (N or less) are active and which contracts are violated. 

The user can choose to perform either compositional or monolithic analysis using a max N fault hypothesis. In compositional analysis, the analysis proceeds in a top down fashion. To prove the top level properties, the properties in the layer directly beneath the top level are used to perform the proof. The analysis proceeds in this manner. Users constrain the maximum number of faults within each layer of the model by specifying the maximum fault hypothesis statement to that layer. If any lower level property failed due to activation of faults, the property verification at the higher level can no longer be trusted because the higher level properties were proved based on the assumption that the direct sub-level contracts are valid. This form of analysis is helpful to see weaknesses in a given layer of the system. 

In monolithic analysis the layers of the model are flattened, which allows a direct correspondence between all faults in the model and their effects on the top level properties. As with compositional analysis, a counterexample shows these N or less active faults. 

\subsubsection{Verification in the Presence of Faults: Probabilistic Analysis} 
Given a probabilistic fault hypothesis, this corresponds to performing analysis with the combinations of faults whose occurrence probability is less than the probability threshold. This is done by inserting assertions that allow those combinations in the Lustre code. If the model checker proves that the safety properties can be violated with any of those combinations, one of such combination will be shown in the counterexample. 

Probabilistic analysis done in this way must utilize the monolithic AGREE option. For compositional probabilistic analysis, see Section~\ref{sec:prob_generate}.

To perform this analysis, it is assumed that the non-hardware faults occur independently and possible combinations of faults are computed and passed to the Lustre model to be checked by the model checker. As seen in Algorithm 1, the computation first removes all faults from consideration that are too unlikely given the probability threshold. The remaining faults are arranged in a priority queue $\mathcal{Q}$ from high to low. Assuming independence in the set of faults, we take a fault with highest probability from the queue (step 5) and attempt to combine the remainder of the faults in $\mathcal{R}$ (step 7). If this combination is lower than the threshold (step 8), then we do not take into consideration this set of faults and instead remove the tail of the remaining faults in $\mathcal{R}$. 
 
\begin{algorithm}[H]
	% \KwData{this text}
	% \KwResult{how to write algorithm with \LaTeX2e }
	$\mathcal{F} = \{\}$ : fault combinations above threshold \;
	$\mathcal{Q}$ : faults, $q_i$, arranged with probability high to low \;
	$\mathcal{R} = \mathcal{Q}$ , with $r \in \mathcal{R}$\;
	\While{$\mathcal{Q} \neq \{\} \land \mathcal{R} \neq \{\}$ }{
		$q =$ removeTopElement($\mathcal{Q}$) \;
		\For{$i=0:|\mathcal{R}|$}{
			$prob = q \times r_i$ \;
			\eIf{prob $<$ threshold}{
				removeTail($\mathcal{R}, j=i:|\mathcal{R}|$)\;
			}{
				add($\{q, r_i\}, \mathcal{Q}$)\;
				add($\{q, r_i\}, \mathcal{F}$)\;
			} % end if else
		} % end for
	} % end while
	\caption{Monolithic Probability Analysis}
	\label{alg:prob_monolithic}
\end{algorithm}
In this calculation, we assume independence among the faults, but in the Safety Annex it is possible to define dependence between faults using a fault propagation statement. After fault combinations are computed using Algorithm~\ref{alg:prob_monolithic}, the triggered dependent HW faults are added to the combination as appropriate. The dependencies are implemented in the \textit{Verify in the Presence of Faults} options for analysis, but not yet implemented in the \textit{Generate Minimal Cut Sets} analysis options.

\subsubsection{Generate Minimal Cut Sets: Max N Analysis}
\label{sec:maxN_generate}
As described in Section~\ref{sec:implementation}, \textit{Generate Minimal Cut Sets} analysis uses the All-MIVCs algorithm to provide a full enumeration of all minimal set of model elements necessary for the proof of each top-level safety property in the model, and then transforms all MIVCs into all minimal cut sets. In Max N analysis, the minimal cut sets are pruned to include only those with at cardinality less or equal to the max N number specified in the fault hypothesis and displayed to the user.

Generate MinCutSet analysis was performed on the Wheel Brake System and results are shown in Table~\ref{tab:wbs_maxN_results}. Notice in Table~\ref{tab:wbs_maxN_results}, the label across the top row refers to the cardinality (C) and how many cut sets of that cardinality. When the analysis is run, the user specifies the value N. This gives cut sets of cardinality \textit{less than or equal to} N. (For the full text of the properties, see Table~\ref{tab:safetyProperties}.)

\begin{center}
\begin{table}[h]
    \begin{tabular}{ | l | l | l | l | l | l | l | l |}
    \hline
    \textbf{Property} & $\bm{c = 1}$ & $\bm{c = 2}$ & $\bm{c = 3}$ & $\bm{c = 4}$ 
		& $\bm{c = 5}$ & $\bm{c = 6}$ & $\bm{c = 7+}$  \\ \hline \hline
    R-0321 & 6 & 0 & 0 & 1& 144&7776 &- \\ \hline
    R-0322 & 32 & 0 & 0 &0 &0 &0 &- \\ \hline
    L-0322 & 32 & 0 & 0 &0 &0 &0 &- \\ \hline
    0323 & 90 & 0 & 0 &0 &0 &0 &- \\ \hline
    0324 & 8 & 3,401 & 6,800 &66,472 & 435,358&1,892,832 &- \\ \hline
    0325-WX & 20 & 0 & 0 &0 &0 & 0&- \\ \hline
    \end{tabular}
    \caption{WBS MinCutSet Analysis Results for Cardinality $c$}
    \label{tab:wbs_maxN_results}
\end{table}
\end{center}



Due to the increasing number of possible fault combinations at $N=6$, the computational time increases quickly. The WBS analysis was only run to $N=6$ for this reason. 

\subsubsection{Generate Minimal Cut Sets: Probabilistic Analysis}
\label{sec:prob_generate}
Both probabilistic analysis and max N analysis use the same minimal cut set generation algorithm, except that in probabilistic analysis, the minimal cut sets are pruned to include only those fault combinations whose probability of simultaneous occurrence exceed the given threshold in the probability hypothesis. Note that with probablistic hypothesis, \textit{Verify in the Presence of Faults} is performed using only monolithic analysis, but generating minimal cut sets is performed using compositional analysis.


The probabilistic analysis for the WBS was given a top level threshold of $1.0 \times 10^{-9}$ as stated in AIR6110. The faults associated with various components were all given probability of occurrence compatible with the discussion in this same document. 

As shown in Table~\ref{tab:wbs_prob_results}, the number of allowable combinations drops considerably when given probabilistic threshold as compared to just fault combinations of certain cardinalities. For example, one contract (inadvertent wheel braking of all wheels) had over a million minimal cut sets produced when looking at it in terms of max N analysis, but after taking probabilities into account, it is seen that only one combination of faults can violate this property. (For the full text of the properties, see Table~\ref{tab:safetyProperties}.)

\begin{center}
\begin{table}[h]
    \begin{tabular}{ | l | l | l | l | l | l | l | l | l |}
    \hline
    \textbf{Property} & $\bm{c = 1}$ & $\bm{c = 2}$ & $\bm{c = 3}$ & $\bm{c = 4}$ 
		& $\bm{c = 5}$ & $\bm{c = 6}$ & $\bm{c = 7}$ & $\bm{c = 8}$  \\ \hline \hline
    R-0321 & 0 & 0 & 0 & 0 & 0 & 0 & 0 & 0 \\ \hline
    R-0322 & 32 & 0 & 0 &0 &0 &0 &0& 0  \\ \hline
    L-0322 & 32 & 0 & 0 & 0 & 0 & 0 & 0 & 0  \\ \hline
    0323 & 90 & 0 & 0 & 0 & 0 & 0 & 0 & 0  \\ \hline
    0324 & 0 & 1 & 0 & 0 & 0 & 0 & 0 & 0  \\ \hline
    0325-WX & 20 & 0 & 0 &0 &0 & 0 & 0 & 0  \\ \hline
    \end{tabular}
    \caption{WBS MinCutSet Results for Probabilistic Analysis}
    \label{tab:wbs_prob_results}
\end{table}
\end{center}

\subsubsection{Results from Generate Minimal Cut Sets}
Results from Generate Minimal Cut Sets analysis can be represented in one of the following forms.
\begin{enumerate}
\item The minimal cut sets can be presented in text form with the total number per property, cardinality of each, and description strings showing the property and fault information. A sample of this output is shown in Figure~\ref{fig:detailedMCS}. 
\begin{figure}[h!]
	\hspace*{-2cm}
	\vspace{-0.1in} 
	\begin{center}
		\includegraphics[scale=0.7]{images/detailedMCS.png}
	\caption{Detailed Output of MinCutSets}
		\label{fig:detailedMCS}
	\end{center}
\end{figure}

\item The minimal cut set information can be presented in tally form. This does not contain the fault information in detail, but instead gives only the tally of cut sets per property. This is useful in large models with many cut sets as it reduces the size of the text file. An example of this output type is seen in Figure~\ref{fig:tallyMCS}.
\begin{figure}[h!]
	\hspace*{-2cm}
	\vspace{-0.1in} 
	\begin{center}
		\includegraphics[scale=0.7]{images/TallyMCS.png}
	\caption{Tally Output of MinCutSets}
		\label{fig:tallyMCS}
	\end{center}
\end{figure}

\item The tool can also generate fault tree and minimal cut set information formatted as input to the SOTERIA tool~\cite{manolios2019model} to produce hierarchical fault trees that are consistent with the system architecture/component verification layers, or flat fault trees consist of minimal cut sets only, both in graphical form. A sample graphical fault tree output from the SOTERIA tool is shown in Figure~\ref{fig:soteriaFT}. The SOTERIA tool is also able to compute the probabilities for the top level event from a given fault tree. However, based on experience with the WBS example, our tool was a more scalable solution as it produces minimal cut sets for more complex systems, also in shorter amount of time. The text format of the minimal cut sets seemed anectodally easier to read than the graphical format for larger systems. 
% minimal cut set information can also be formatted as input to the SOTERIA tool \janet{[new soteria reference]} to display. 
%This
%\janet{The SOTERIA} tool can \janet{produce fault trees in }
%use the auto generated ocaml file to produce optimized fault trees in a graphical format instead of textual format. (See Section on SOTERIA for more information.) A sample output is shown in Figure~\ref{fig:soteriaMCS}. This \textit{.ml} file can be given as input to %SOTERIA to produce the optimized fault tree shown in Figure~\ref{fig:}. 
\begin{comment}
\begin{figure}[h!]
	\hspace*{-2cm}
	\vspace{-0.1in} 
	\begin{center}
		\includegraphics[scale=0.5]{images/soteriaMCS.png}
	\caption{SOTERIA Output of MinCutSets}
		\label{fig:soteriaMCS}
	\end{center}
\end{figure}
\end{comment}

\begin{figure}[h!]
	\hspace*{-2cm}
	\vspace{-0.1in} 
	\begin{center}
		\includegraphics[scale=0.25]{images/Soteria_FT.png}
		\caption{Example SOTERIA Fault Tree}
		\label{fig:soteriaFT}
	\end{center}
\end{figure}

\end{enumerate}

\subsubsection{Use of Analysis Results to Drive Design Change}
We use a single top level requirement of the WBS: S18-WBS-0323 (Never indadvertent braking with all wheels locked to illustrate how Safety Annex can be used to detect design flaws and how faults can affect the behavior of the system). This safety property description can be found in detail in Section \ref{sec:fault_modeling}. Upon running max $n$ compositional fault analysis with $n = 1$, this particular fault was shown to be a single point of failure for this safety property. A counterexample is shown in Figure \ref{fig:counterexample} showing the active fault on the pedal sensor. 

\begin{figure}[h!]
	%\vspace{-0.2in}
	\begin{center}
		\includegraphics[width=\textwidth]{images/counterexample.png}
	\end{center}
	\vspace{-0.3in}
	\caption{AGREE counterexample for inadvertent braking safety property}
	\label{fig:counterexample}
	%\vspace{-0.2in}
\end{figure} 

Depending on the goals of the system, the architecture currently modeled, and the mitigation strategies that are desired, various strategies are possible to mitigate the problem.

\begin{itemize}
\item Possible mitigation strategy 1: Monitor system can be added for the sensor: A monitor sub-component can be modeled in which it accesses the mechanical pedal as well as the signal from the sensor. If the monitor finds discrepancies between these values, it can send an indication of invalid sensor value to the top level of the system. In terms of the modeling, this would require a change to the behavioral contracts which use the sensor value. This validity would be taken into account through the means of $valid \land pedal\_sensor\_value$. 
%In the real system however, this mitigation would need to be taken into account. Whether this is a flag to the pilot who can then override the electrical system and switch to a different mode or perform some other action to mitigate the failed sensor must be discussed and implemented. 

\item Possible mitigation strategy 2: Redundancy can be added to the sensor: A sensor subsystem can be modeled which contains 3 or more sensors. The overall output from the sensor system may utilize a voting scheme to determine validity of sensor reading. There are multiple voting schemes that are possible, one of which is a majority voting (e.g. one sensor fails, the other two take majority vote and the correct value is passed). 
When three sensors are present, this mitigates the single point of failure problem. New behavioral contracts are added to the sensor system to model the behavior of redundancy and voting. 
\end{itemize}

In the case of the pedal sensor in the WBS, the latter of the two strategies outlined above was implemented. A sensor system was added to the model which held three pedal sensors. The output of this subsystem was constrained using a majority voting scheme. Upon subsequent runs of the analysis (regardless which type of run was used), resilience was confirmed in the system regarding the failure of a single pedal sensor. Figure \ref{fig:sensorsystem} outlines these architectural changes that were made in the model.

\begin{figure}[h!]
	%\vspace{-0.2in}
	\begin{center}
		\includegraphics[width=\textwidth]{images/sensorsystem.png}
	\end{center}
	\vspace{-0.3in}
	\caption{Changes in the architectural model for fault mitigation}
	\label{fig:sensorsystem}
	%\vspace{-0.2in}
\end{figure}

As can be seen through this single example, a system as large as the WBS would benefit from many iterations of this process. Furthermore, if the model is changed even slightly on the system development side, it would automatically be seen from the safety analysis perspective and any negative outcomes would be shown upon subsequent analysis runs. This effectively eliminates any miscommunications between the system development and analysis teams and creates a new safeguard regarding model changes. 

For more information on types of fault models that can be created as well as details on analysis results, see the users guide located in the GitHub repository \cite{SAGithub}. This repository also contains all models used in this project. 




\section{Related Work}
\label{sec:related_work}

We view model-based safety analysis as an extension of model-based systems engineering. Our 
goal is to extend an existing system model to add information needed for safety analysis.
Model-based safety analysis approaches have been developed for a variety of modeling languages, 
including SysML~\cite{friedenthal2014practical, helle2012automatic,mhenni2014automatic}, AADL~\cite{AADL_Standard, EMV2}, SLIM~\cite{10.1007/978-3-642-04468-7_15}, Simulink~\cite{MathWorks, Joshi05:SafeComp}, and others. 
Each language has a targeted domain of application and contains different levels of formalism.
For example, SysML is normally used to describe graphical system model in the early stages of 
development to explore requirement, use cases, and design trade-offs, while AADL provides a 
more rigorous system description and run-time semantics that better supports analysis and 
implementation.  Due to the formal model checking approach we have used in the safety annex, 
we chose to extend a modeling language with well-defined semantics with a focus on modeling 
real-time embedded systems.

A model-based approach for safety analysis was proposed by Joshi et. al in \cite{Joshi05:Dasc, Joshi05:SafeComp, Joshi07:Hase}.  In this approach, a \gls{sasm} is the central artifact in the safety analysis process, and traditional safety analysis artifacts, such as fault trees, are automatically generated by tools that analyze the SASM.

The contents and structure of the \gls{sasm} differ significantly across different conceptions of \gls{mbsa}.  We can draw distinctions between approaches along several different axes.  The first is whether they propagate faults explicitly through user-defined propagations, which we call \gls{flm} or through existing behavioral modeling, which we call \gls{fem}.  The next is whether models and notations are {\em purpose-built} for safety analysis vs. those that extend {\em existing system models} (ESM).

For \gls{fem} approaches, there are several additional dimensions.  One dimension involves whether {\em causal} or {\em non-causal} models are allowed.  Non-causal models allow simultaneous (in time) bi-directional %failure
error propagations, which allow more natural expression of some failure types (e.g. reverse flow within segments of a pipe), but are more difficult to analyze.  A final dimension involves whether analysis is {\em compositional} across layers of hierarchically-composed systems or {\em monolithic}.  Our approach is an extension of \gls{aadl} (\gls{esm}), causal, compositional, mixed \gls{flm}/\gls{fem} approach.

Tools such as the \gls{aadl} Error Model Annex, Version 2 (EMV2)~\cite{EMV2}, HiP-HOPS for EAST-ADL~\cite{CHEN201391}, and Ansys Medini~\cite{ansys} are {\em \gls{flm}}-based {\em \gls{esm}} approaches.  As previously discussed, given many possible faults, these propagation relationships require substantial user effort and become more complex.  In addition, it becomes the analyst's responsibility to determine whether faults can propagate; missing propagations lead to unsound analyses.  In the safety annex, propagations occur through system behaviors (defined by the nominal contracts) with no additional user effort.

\begin{figure}[h!]
	\begin{centering}
		\includegraphics[width=\textwidth]{images/relatedwork.jpg}
		\caption{Related MBSA Tools and Methods}
		\label{fig:related}
	\end{centering}
\end{figure}

Figure~\ref{fig:related} shows a reference table listing a few of the related work tools we describe in the remainder of this section. The figure highlights important features of the support provided. Closely related to our work is the model-based safety assessment toolset called COMPASS (Correctness, Modeling project and Performance of Aerospace Systems)~\cite{10.1007/978-3-642-04468-7_15}.  COMPASS is a mixed {\em \gls{flm}/\gls{fem}}-based, {\em causal} tool suite that uses the SLIM language, which is based on a subset of AADL, for its input models~\cite{5185388, criticalembeddedsystems}. In SLIM, a nominal system model and the error model are developed separately and then transformed into an extended system model.  This extended model is automatically translated into input models for the NuSMV model checker~\cite{Cimatti2000, NuSMV}, MRMC (Markov Reward Model Checker)~\cite{Katoen:2005:MRM:1114692.1115230, MRMC}, and RAT (Requirements Analysis Tool)~\cite{RAT}. The safety analysis tool xSAP~\cite{DBLP:conf/tacas/BittnerBCCGGMMZ16} can be invoked in order to generate safety analysis artifacts such as fault trees and FMEA tables~\cite{compass30toolset}.  COMPASS is an impressive tool suite, but some of the features that make AADL suitable for SW/HW architecture specification: event and event-data ports, threads, and processes, appear to be missing, which means that the SLIM language may not be suitable as a general system design notation (\gls{esm}).

SmartIFlow~\cite{info17:HaLuHo} is a {\em \gls{fem}}-based, {\em purpose-built}, {\em monolithic} {\em non-causal} safety analysis tool that describes components and their interactions using finite state machines and events. Verification is done through an explicit state model checker which returns sets of counterexamples for safety requirements in the presence of failures.  SmartIFlow allows {\em non-causal} models containing simultaneous (in time) bi-directional %failure
error propagations.  On the other hand, the tools do not yet appear to scale to industrial-sized problems, as mentioned by the authors~\cite{info17:HaLuHo}: ``As current experience is based on models with limited size, there is still a long way to go to make this approach ready for application in an industrial context.''

The Safety Analysis and Modeling Language (SAML)~\cite{Gudemann:2010:FQQ:1909626.1909813} is a {\em FEM}-based, {\em purpose-built}, {\em monolithic} {\em causal} safety analysis language.  System models constructed in SAML can be used for both qualitative and quantitative analyses. It allows for the combination of discrete probability distributions and non-determinism. The SAML model can be automatically imported into several analysis tools like NuSMV~\cite{Cimatti2000}, PRISM (Probabilistic Symbolic Model Checker)~\cite{CAV2011:KwNoPa}, or the MRMC probabilistic model checker~\cite{Katoen:2005:MRM:1114692.1115230}. 

AltaRica~\cite{PROSVIRNOVA2013127,BieberERTS2018} is a {\em \gls{fem}}-based, {\em purpose-built}, {\em monolithic} safety analysis language with several dialects.  There is one dialect of AltaRica which uses dataflow ({\em causal}) semantics, while the most recent language update (AltaRica 3.0) uses non-causal semantics.  The dataflow dialect has substantial tool support, including the commercial Cecilia OCAS tool from Dassault~\cite{Bieber04safetyassessment}.  For this dialect the Safety assessment, fault tree generation, and functional verification can be performed with the aid of NuSMV model checking~\cite{symbAltaRica}. Failure states are defined throughout the system and flow variables are updated through the use of assertions~\cite{BieberERTS2018}.  AltaRica 3.0 has support for simulation and Markov model generation through the OpenAltaRica (www.openaltarica.fr) tool suite.

Formal verification tools based on model checking have been used to automate the generation of safety artifacts~\cite{symbAltaRica,10.1007/978-3-540-75596-8-13, DBLP:conf/tacas/BittnerBCCGGMMZ16}. This approach has limitations in terms of scalability and readability of the fault trees generated. Work has been done towards mitigating these limitations by the scalable generation of readable fault trees~\cite{10.1007/978-3-319-11936-6-7}.






\section{Conclusion}
\label{sec:conclusion}
We have developed an extension to the \gls{aadl} language with tool support for formal analysis of system safety properties in the presence of faults. The nominal model is extended with fault definitions, which allows safety analysis and system implementation to be driven from a single common model. The use of formal methods supports comprehensive exploration on the effect of faulty component behaviors on the system level failure condition without the need to add separate propagation specifications to the model. During the development of this approach we worked closely with safety engineers to ensure that the needs of the analysts are supported. This approach was illustrated through the use case of an aircraft system, but can be applied on the development of critical systems in multiple domains. 

The contributions described in this paper are as follows:

\begin{itemize}
\renewcommand{\labelitemi}{\textbullet}
		\item close integration of behavioral fault analysis into \gls{aadl}, which allows close connection between system and safety analysis and system generation from the model,
		\item support for behavioral specification of faults and their  implicit propagation through behavioral relationships in the model,
		\item additional support to capture binding relationships between hardware and software and logical and physical communications, %and
		\item the use of formal methods to automatically verify safety properties in the presence of faults and comprehensively enumerate all applicable fault combinations leading to failure conditions under quantitative objectives as part of the safety assessment process, and
		\item guidance on integration into a traditional safety analysis process.
\end{itemize}

Future work includes compilation of minimal cut sets into graphical fault tree format, expanding the user interface to provide ease in fault model creation, and transforming the counterexample into a sequence flow showing how the system changes as faults are activated. The research presented in this paper, as well as the contributions of future work, all serve to support the safety assessment process. These contributions do not encompass all of the assessment process, but instead aim to provide automated and comprehensive analysis and also to generate evidence for the assessment process.



\vspace{2 mm}
\noindent {\bf Acknowledgments.} This research was funded by NASA contract NNL16AB07T and the University of Minnesota College of Science and Engineering Graduate Fellowship.



\printglossary[type=\acronymtype]
\printnoidxglossary

\bibliographystyle{abbrv}
\bibliography{biblio}
%\vspace{-7.25cm}
% This ~ seems to fix an odd bibliography alignment issue
~

%\ifdefined\TECHREPORT
%\appendix
%
%\section{Appendix: Proof of Equivalence}
%\input{appendix}
%\fi

%\section{Appendix: GPCA CENTA Model}
%\label{appendix:gpcacenta}
%\begin{figure}[!ht]
%\begin{center}
%\includegraphics[scale=0.6]{images/sampled_pca.PNG} %[trim = 0 2 0 0, clip=true]{Comp}
%\caption{GPCA AGREE Properties modeled as a Timed Automata} \label{fig:samplepca}
%\end{center}
%\end{figure}

%\balancecolumns

\end{document} 