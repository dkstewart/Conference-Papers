\section{Fault Modeling with the Safety Annex}
\label{sec:fault_modeling}

To demonstrate the fault modeling capabilities of the Safety Annex we will use the Wheel Brake System (WBS) described in AIR6110~\cite{AIR6110}.  This system is a well-known example that has been used as a case study for safety analysis, formal verification, and contract based design~\cite{DBLP:conf/cav/BozzanoCPJKPRT15, 10.1007/978-3-319-11936-6-7, CAV2015:BoCiGrMa, Joshi05:SafeComp}. The preliminary work for the safety annex was based on a simple model of the WBS~\cite{Stewart17:IMBSA}. To demonstrate a more complex fault modeling process, we constructed a functionally and structurally equivalent AADL version of the more complex WBS NuSMV/xSAP models~\cite{DBLP:conf/cav/BozzanoCPJKPRT15}, as illustrated in Figure \ref{fig:inadvertent_braking}.     

\begin{figure*}[t]
	\centering
	\includegraphics[trim=0 9 0 5,clip,width=.7\dimexpr\textwidth-2cm\relax]{images/wbs_arch4_diagram.pdf}
	\caption{Wheel Brake System}
	\label{fig:wbs}
\end{figure*} 


The WBS is composed of two main parts: the Line Replaceable Unit control system and the electro-mechanical physical system.
The control system electronically controls the physical system and contains a redundant
channel of the Braking System Control Unit (BSCU) in case a detectable fault occurs in the active channel.
 It also commands antiskid braking. % in case of skidding on the ground. 
 The physical system consists of the hydraulic circuits running from hydraulic pumps to wheel brakes as well as valves that control the hydraulic fluid flow. This system provides braking force to each of the eight wheels of the aircraft. The wheels are all mechanically braked in pairs (one pair per landing gear). For simplicity, Figure~\ref{fig:wbs} displays only two of the eight wheels. 

There are three operating modes in the WBS model:

\begin{itemize}
	\renewcommand{\labelitemi}{\textbullet}
	\item In \textit{normal} mode, the system is composed of a \textit{green} hydraulic pump and one meter valve per each of the eight wheels. Each of the meter valves are controlled through electronic commands coming from the active channel of the BSCU. These signals provide braking and antiskid commands for each wheel. The braking command is determined through a sensor on the pedal and the antiskid command is determined by the \textit{Wheel Sensors}. 
	\item In \textit{alternate} mode, the system is composed of a \textit{blue} hydraulic pump, four meter valves, and four antiskid shutoff valves, one for each landing gear. The meter valves are mechanically commanded through the pilot pedal corresponding to each landing gear. There are two ways the system can change into alternate mode. The selector can choose the blue circuit when the BSCU sends a system invalid signal, or the system can switch to the blue circuit when the selector detects lack of pressure in the green circuit.
	%There are two ways the system can change into alternate mode. Either the selector detects lack of pressure in the green circuit and switches to the blue circuit or the BSCU sends a validity signal to the selector in which case the selector chooses the blue line.
	\item In \textit{emergency} mode, the system mode is entered if the \textit{blue} hydraulic pump fails. The accumulator pump has a reserve of pressurized hydraulic fluid and will supply this to the blue circuit in emergency mode. 
\end{itemize}

The WBS architecture model in AADL contains 30 different kinds of components, 169 component instances, and a model depth of 5 hierarchical levels. 


%If the BSCU channel becomes invalid, the shutoff valve closes and we move into alternate mode. Once this system switches into alternate mode, it does not return to normal operation mode.

%There are three operating modes in the WBS model. In \textit{normal} mode, the system uses the \textit{green} hydraulic circuit. The normal system is composed of the green hydraulic pump and one meter valve per each of the eight wheels. Each of the meter valves are controlled through electronic commands coming from the active channel of the BSCU. These signals provide braking and antiskid commands for each wheel. The braking command is determined through a sensor on the pilot pedal position and is labeled as \textit{Left/Right Pedal Sensor} in Figure~\ref{fig:wbs} and the antiskid command is determined by the \textit{Wheel Sensors}. 

%In \textit{alternate} mode, the system uses the \textit{blue} hydraulic circuit. The alternate system is composed of the blue hydraulic pump, four meter valves, and four antiskid shutoff valves: one for each landing gear. The meter valves are mechanically commanded through the pilot pedal corresponding to each landing gear. If the system detects lack of pressure in the green circuit, the BSCU channel commands the selector valve to switch to the blue circuit. 
%If the BSCU channel becomes invalid, the shutoff valve closes and we move into alternate mode. Once this system switches into alternate mode, it does not return to normal operation mode.

%The last mode of operation of the WBS is the \textit{emergency} mode. This mode is entered if the blue hydraulic pump fails. The accumulator pump has a reserve of pressurized hydraulic fluid and will supply this to the blue circuit in emergency mode.

%The model contains 30 different kinds of components, 169 component instances, a model depth of 5 hierarchical levels.  The model includes one top-level assumption and  11 top-level system properties, with 113 guarantees allocated to subsystems.  There are a total of 33 different fault types and 141 fault instances within the model.  The large number of fault instances is due to the redundancy in the system design and its replication to control 8 wheels.

The behavioral model is encoded using the AGREE annex and the behavior is based on descriptions found in AIR6110. The top level system properties are given by the requirements and safety objectives in AIR6110. All of the subcomponent contracts support these system safety objectives through the use of assumptions on component input and guarantees on the output. The WBS behavioral model in AGREE annex includes one top-level assumption and  11 top-level system properties, with 113 guarantees allocated to subsystems.  

An example system safety property is to ensure that there is no inadvertent braking of any of the wheels. This is based on a failure condition described in AIR6110 is \textit{Inadvertent wheel braking on one wheel during takeoff shall be less than %1E-9 
$1.0\times 10^{-9}$ per takeoff}. 
Inadvertent braking means that braking force is applied at the wheel but the pilot has not pressed the brake pedal.  In addition, the inadvertent braking requires that power and hydraulic pressure are both present, the plane is not stopped, and the wheel is rolling (not skidding). The property is stated in AGREE such that inadvertent braking does \textit{not} occur, as shown in Figure \ref{fig:inadvertent_braking}. 

\begin{figure}[h!]
	%\vspace{-0.2in}
	\begin{center}
	\includegraphics[width=.9\columnwidth]{images/inadvertent_braking.png}
	\end{center}
	%\vspace{-0.3in}
	\caption{AGREE Contract for Top Level Property: Inadvertent Braking}
	\label{fig:inadvertent_braking}
	\vspace{-0.2in}
\end{figure}


\subsection{Component Fault Modeling}

The usage of the terms error, failure, and fault are defined in ARP4754A and are described here for clarity
%ease of understanding
~\cite{SAE:ARP4754A}. An \textit{error} is a mistake made in implementation, design, or requirements. A \textit{fault} is the manifestation of an error and a \textit{failure} is an event that occurs when the delivered service of a system deviates from correct behavior. If a fault is activated under the right circumstances, that fault can lead to a failure. The terminology used in EMV2 differs slightly for an error: an error is a corrupted state caused by a fault. The error propagates through a system and can  manifest as a failure. In this paper we use the ARP4754A terminology with the added definition of \textit{error propagation} as used in EMV2. An error is a mistake made in design or code and an error propagation is the propagation of the corrupted state caused by an active fault. 

The Safety Annex is used to add possible faulty behaviors to a component model. Within the AADL component instance model, an annex is added which contain the fault definitions for the given component. The flexibility of the fault definitions allows the user to define numerous types of fault \textit{nodes} by utilizing the AGREE node syntax. A library of common fault nodes has been written and is available in the project GitHub repository~\cite{SAGithub}. Examples of such faults include valves being stuck open or closed, output of a software component being nondeterministic, or power being cut off.  When the fault analysis requires fault definitions that are more complex, these nodes can easily be created and used in the model. 
%\janet{Since AADL models is an abstract representation of the system, expected and faulty behaviors from the operators could also be modeled in AGREE and Safety Annex.}

When a fault is activated by its specified triggering conditions, it modifies the output of the component. This faulty behavior may lead to the violation of the contracts of other components in the system, including assumptions of downstream components. The impact of a fault is computed by the AGREE model checker when the safety analysis is run on the fault model. 

The majority of faults that are connected to outputs of components are known as \textit{symmetric}. That is, whatever components receive this faulty output will receive the same faulty output value. Thus, this output is seen symmetrically. An alternative fault type is \textit{asymmetric}. This pertains to a component with a 1-n output: one output which is sent to many receiving components. This fault can present itself differently to the receiving components. For instance, in a boolean setting, one component might see a true value and the rest may see false. This type of fault is modeled using the keyword \textit{asymmetric}. For more information on fault definitions and modeling possibilities, we refer readers to the Safety Annex Users Guide\cite{SAGithub}. %\danielle{This might be all we need for mentioning asym faults...}

As an illustration of fault modeling using the Safety Annex, we look at one of the components relevant to the inadvertent braking property: the brake pedal. When the mechanical pedal is pressed, a sensor reads this information and passes an electronic signal to the BSCU that then commands hydraulic pressure to the wheels. 

\begin{comment}
%Figure~\ref{fig:sensor} 
Figure~\ref{fig:sensor} shows the AADL pedal sensor component with a contract for its nominal behavior. The sensor has only one input, the mechanical pedal position, and one output, the electrical pedal position. 
A property that governs the behavior of the component is that the mechanical position should always equal the electronic position. (The expression \textit{true $\rightarrow$ property} in AGREE is true in the initial state and then afterwards it is only true if property holds.)

\begin{figure}[h!]
	%\hspace*{-2cm}
	%\vspace{-0.55in} 
	\begin{center}
		\includegraphics[trim=0 640 -10 70,clip,width=.8\dimexpr\textwidth-2cm\relax]{images/system_sensor.pdf}
		%\vspace{-0.3in}
		\caption{An AADL System Type: The Pedal Sensor}
		\label{fig:sensor}
	\end{center}
	\vspace{-0.2in}
\end{figure}
\end{comment}

One possible failure for this sensor is inversion of its output value. This fault can be triggered with probability $5.0\times 10^{-6}$ as described in AIR6110 (in reality, the component failure probability is 
collected from hardware specification sheets).  
The Safety Annex definition for this fault is shown in Figure~\ref{fig:sensorFault}. Fault behavior is defined through the use of a fault node called \textit{inverted\_fail}.  When the fault is triggered, the nominal output of the component (\textit{elec\_pedal\_position}) is replaced with its failure value (\textit{val\_out}). 

\begin{figure}[h!]
	%\hspace*{-2cm}
	%\vspace{-0.5in} 
	\begin{center}
		\includegraphics[width=.9\columnwidth]{images/safetyannex_sensorfault.png}
		%\vspace{-0.2in}
		\caption{The Safety Annex for the Pedal Sensor}
		\label{fig:sensorFault}
	\end{center}
	\vspace{-0.1in}
\end{figure}

The WBS fault model expressed in the Safety Annex contains a total of 33 different fault types and 141 fault instances. The large number of fault instances is due to the redundancy in the system design and its replication to control 8 wheels.

\subsection{Implicit Error Propagation}
In the Safety Annex approach, faults are captured as faulty behaviors that augment the system behavioral model in AGREE contracts. Unlike AADL EMV2, no explicit %fault
error propagation is necessary since the faulty behavior itself propagates through the nominal behavior contracts of the non-failed system components. The effects of any triggered fault are manifested through analysis of the AGREE contracts.  

%On the contrary, in the AADL Error Model Annex, Version 2 (EMV2)~\cite{EMV2} approach, all errors must be explicitly propagated through each component (by applying fault types on each of the output ports) in order for a component to have an impact on the rest of the system. To illustrate the key differences between implicit %failure
%error propagation provided in the Safety Annex and the explicit %failure 
%error propagation provided in EMV2, we use a simplified behavioral flow from the WBS example using code fragments from EMV2, AGREE, and the Safety Annex. 

\begin{figure*}[t]
	%\hspace*{-2cm}
	%\vspace{-0.19in}
	\centering
	\includegraphics[width=0.7\textwidth]{images/Comparison_with_EMV2.png}
	%\vspace{-0.3in}
	\caption{Differences between Safety Annex and EMV2}
	\label{fig:comparison_with_EMV2}
	%\vspace{-0.2in}
\end{figure*} 

To illustrate key differences between implicit error propagation in the Safety Annex and explicit propagation in  EMV2, we use a simple example derived from the WBS, shown in Figure~\ref{fig:comparison_with_EMV2}. 
In this simplified WBS system, the physical signal from the Pedal component is detected by the Sensor and the pedal position value is passed to the Braking System Control Unit (BSCU) components.  The BSCU generates a pressure command to the Valve component which applies hydraulic brake pressure to the Wheels. 

In the EMV2 approach (top half of Figure~\ref{fig:comparison_with_EMV2}), the ``NoService'' fault is explicitly propagated through all of the components. These fault types are essentially tokens that do not capture any analyzable behavior. At the system level, analysis tools supporting the EMV2 annex can aggregate the propagation information from different components to compose an overall fault flow diagram or fault tree. 

The Safety Annex is used to add failure behaviors to the Sensor and Valve components and to specify the fault hypothesis (in this case, a single fault). 
When a fault condition is triggered in the analysis, the output behavior of that component is produced by its failure contract instead of its nominal contract.  For our simplified example, if the Sensor component fault condition is triggered, the result is a ``stuck at zero'' error. The contract of the BSCU receives a zero input and proceeds as if the pedal has not been pressed. This will cause the top level system contract to fail.

\begin{comment}
%\subsection{Comparison to the AADL Error Annex}

The AADL language has previously been extended to provide some fault modeling and analysis capabilities using its Error Model Annex, Version 2 (EMV2)~\cite{EMV2}.  EMV2 focuses on injection and propagation of discrete faults for generation of fault trees, rather than on analysis of system behavior in the presence of faults. 
To illustrate some of the key differences between our approach and the EMV2 approach, Figure~\ref{fig:comparison_with_EMV2} shows a simplified  behavioral flow with code fragments from EMV2, AGREE, and the Safety Annex. 

\begin{figure}[t]
	\vspace{-0.45in}
	\centering
	\includegraphics[trim=0 9 0 5,clip,width=.8\dimexpr\textwidth-2cm\relax]{images/Comparison_with_EMV2.pdf}
	%\vspace{0.4in}
	\caption{Differences between Safety Annex and EMV2}
	\label{fig:comparison_with_EMV2}
\end{figure} 

In this simplified WBS system, the physical signal from the Pedal component is detected by the Sensor, and the pedal position value is passed to the Braking System Control Unit (BSCU) components.  The BSCU generates a pressure command to the Valve component which applies hydraulic brake pressure to the Wheels. 

In the EMV2 approach (top half of Figure~\ref{fig:comparison_with_EMV2}), all errors must be explicitly propagated through each component (by applying fault types on each of the output ports) in order for a component to have an impact on the rest of the system. In the example, the ``NoService'' fault is explicitly allowed by the EMV2 declarations to propagate through all of the components.  These fault types are essentially tokens that do not capture any analyzable behavior.  At the system level, analysis tools supporting the EMV2 annex can aggregate the fault flow and propagation information from different components to compose an overall fault flow diagram or fault tree.

In the Safety Annex approach (bottom half of Figure~\ref{fig:comparison_with_EMV2}), faults augment the system behavioral model through the AGREE contracts.  When a fault is triggered, the output behavior of the Sensor component is modified, in this case resulting a ``stuck at zero'' error. The behavior of the BSCU receives a zero input and proceeds as if the pedal has not been pressed. This will cause the top level system contract to fail: {\em pedal pressed implies brake pressure output is positive}. No explicit propagation is necessary since the faulty behavior propagates through the system just as in the nominal system model. The system and component failures are manifested through analysis of the AGREE contracts. 
\end{comment}

\subsection{Explicit %Failure 
	Error Propagation} 
%\janet{consider shortening this section}
%\janet{consider mentioning byzantine fault modeling}
%Faults
Failures in hardware (HW) components can trigger behavioral faults in the system components that depend on them. For example, a CPU %fault
Failure may trigger faulty behavior in the threads bound to that CPU. In addition, a %fault
failure in one HW component may trigger %faults
failure in other HW components located nearby, such as overheating, fire, or explosion
in the containment location. 
The Safety Annex provides the capability to explicitly model the impact of hardware %faults
failures on other faults, behavioral or non behavioral. The explicit propagation to non behavioral faults is similar to that provided in EMV2.

To better model %HW dependent faults 
faults at the system level dependent on HW failures, a fault model element is introduced called a \textit{hardware fault}. Users are not required to specify behavioral effects for the HW faults, nor are data ports necessary on which to apply the fault definition. 
\begin{comment}
An example of a model component fault declaration is shown \janet{in Figure~\ref{fig:hwFault}.} %below:
\begin{figure}[h!]
	\vspace{-0.1in}
	\begin{center}
	\includegraphics[width=.4\textwidth]{images/hw_fault2.png}
	\end{center}
	\vspace{-0.1in}
	\caption{Hardware Fault Definition}
	\label{fig:hwFault}
	%\vspace{-0.2in}
	%\vspace{-0.1in}
\end{figure}
\end{comment}
Users specify dependencies between the HW component faults and faults that are defined in other components, either HW or SW. The hardware fault then acts as a trigger for dependent faults. This allows a simple propagation from the faulty HW component to the SW components that rely on it, affecting the behavior on the outputs of the affected SW components. 
%\janet{An example is shown in Figure~\ref{fig:hwFaultProp}.}

\begin{comment}
\danielle{Consider leaving everything above and cutting what is left for explicit faults. It is just an example of the HW fault and some figures.}
%As an example, we will look yet again at the WBS. 
In the WBS example, assume that both the green and blue hydraulic pumps are located in the same compartment in the aircraft and an explosion in this compartment rendered both pumps inoperable.
%An accident took place and the green (normal) hydraulic pump took the force of an explosion. When the green hydraulic pump exploded, the pump shrapnel flew into the blue pump and it became from then on unoperable. 
The HW fault definition can be modeled first in the green hydraulic pump component as shown in Figure~\ref{fig:hwFault}. The activation of this fault triggers the activation of related faults as seen in the \textit{propagate\_to} statement shown below. % in Figure~\ref{fig:hwFaultProp}. 
Notice that these pumps need not be connected through a data port in order to specify this propagation. %Furthermore, the probability of the HW fault activation can be specified. 


\begin{figure}[h!]
	\vspace{-0.1in}
	\begin{center}
		\includegraphics[width=.5\textwidth]{images/hw_prop_stmt.png}
	\end{center}
	\vspace{-0.1in}
	\caption{Hardware Fault Propagation Statement}
	\label{fig:hwFaultProp}
	%\vspace{-0.2in}
	%\vspace{-0.1in}
\end{figure}
\end{comment}

%The fault dependencies are specified in the system implementation where the system configuration that causes the dependencies becomes clear (e.g., binding between SW and HW components, co-location of HW components).

%This is because fault propagations are typically tied to the way components are connected or bound together; this information may not be available when faults are being specified for individual components. Having fault propagations specified outside of a component’s fault statements also makes it easier to reuse the component in different systems. 



\subsection{Fault Analysis Statements}
%An annotation in the AADL model determines the fault analysis statement (also referred to in this report as the fault hypothesis). This may specify either a maximum number of faults that can be active at any point in execution:
The Safety Annex also allows the user to specify constraints on the occurence of faults in the system.  
\begin{itemize}
\item The \emph{max fault hypothesis} specifies the maximum number of faults that can be active at any point in the analysis.  This is analogous to restricting the cutset to a specified maximum number of terms in the fault tree analysis in a traditional safety analysis.  
\item The \emph{probabilistic fault hypothesis} specifieds that only faults whose probability of simultaneous occurrence is above the given threshold should be onsidered.  This is analogous to restricting the cutsets to only those whose probability is above the specified value. 
\end{itemize}

%The fault analysis statement (also referred to as the fault hypothesis) resides in the AADL system implementation that is selected for verification. This may specify either a maximum number of faults that can be active at any point in execution.

\begin{comment}
\begin{figure}[h!]
	\vspace{-0.1in}
	%\begin{center}
		\includegraphics[width=0.4\textwidth]{images/hypothesisMaxN.png}
	%\end{center}
	\vspace{-0.1in}
	%%\caption{Max N Faults Analysis Statement}
	\label{fig:hypothesisMaxN}
\end{figure}
or that the only faults to be considered are those whose probability of simultaneous occurrence is above some probability threshold: 

\begin{figure}[h!]
	\vspace{-0.1in}
	%\begin{center}
		\includegraphics[width=0.5\textwidth]{images/hypothesisProb.png}
	%\end{center}
	\vspace{-0.1in}
	%\caption{Probability Analysis Statement}
	%\label{fig:hypothesisProb}
\end{figure}
\end{comment}
\noindent
In the former case, we assert that the sum of the true {\em fault\_\_trigger} variables is at or below some integer threshold.  In the latter, we determine all combinations of faults whose probabilities are above the specified probability threshold, and describe this as a proposition over {\em fault\_\_trigger} variables. 
%
\subsection{Fault Activation}
With the introduction of dependent faults, active faults are divided into two categories: independently active (activated by its own triggering event) and dependently active (activated when the faults they depend on become active). The top level fault hypothesis applies to independently active faults. Faulty behaviors augment nominal behaviors whenever their corresponding faults are active (either independently active or dependently active). 
%\janet{In addition, there is fault activation statement allowing users to access the activation status of specific faults so they can be used in property specifications, such as to describe a failing or non failing component.}









