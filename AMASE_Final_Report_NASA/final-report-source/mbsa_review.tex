\subsection{Comparison with Proposed MBSA Appendix to ARP4761A}
\label{sec:mbsa_appendix_review}

ARP4754A, the Guidelines for Development of Civil Aircraft and Systems~\cite{SAE:ARP4754A}, provides guidance on applying development assurance at each hierarchical level throughout the development life cycle of highly-integrated/complex aircraft systems. ARP4761, the Guidelines and Methods for Conducting Safety Assessment Process on Civil Airborne Systems and Equipment~\cite{SAE:ARP4761},  identifies a systematic means to show compliance. A Model Based Safety Analysis (MBSA) appendix has been drafted to the upcoming revision of ARP4761 to provide concepts and processes with Model Based Safety Analysis.

We have reviewed the draft appendix and found that our approach is consistent with the MBSA appendix in the following ways:

\begin{itemize}
	\item The common goal is to use MBSA for an equivalent analysis to the traditional safety analysis methods (e.g., Fault Trees) to support safety assessment processes.
	\item Both use an analytical model of the system to capture failure propagation. In the model, system architecture, nominal and faulty functional behaviors are captured. The model evolves as the system design evolves.
	\item Both use software application/tools to perform analysis on the model and generate outputs (e.g., failure sequences, minimal cut sets that result in the failure condition under analysis). The MBSA appendix also mentioned that model checking can be used to perform an exhaustive exploration of the state space of the model.
	\item Outputs generated from the analysis are to be compared to qualitative and/or quantitative objectives and requirements as part of the safety assessment process. Furthermore, the outputs drive evolution of system design. 
\end{itemize}

Our approach goes beyond what is envisioned in the MBSA appendix in the following ways:

\begin{itemize}
	\item The MBSA Appendix is not advocating a single unified model used by both system development and safety assessment activities. The model is safety specific and driven by the types of safety assessment to be conducted. However, the initial safety model may be derived from the system design model, and may be closer to the design at the lower levels of the design process.
	\item In the MBSA Appendix, the failure propagation modeling focuses on the inside internal flows in the components, which is similar to the bottom-up method in Failure Modes and Effects Analysis. Different components are connected by inputs and outputs, and no behavioral constraints are specified on data entering and exiting components. This leaves inter-component propagation to be explored by the analysis.
\end{itemize}

In summary, our approach provides a new way to do safety analysis. It uses an unified model that is shared by system development and safety assessment. The model captures architecture and behavioral information for propagation within components and between components. It is a property driven approach that is consistent between system verification and safety analysis.




































